\section{Installation, update, use}

{\setbeamercolor{background canvas}{bg=LemonChiffon}
\begin{frame}
\frametitle{}
{\fontsize{50}{60}\selectfont Installation\\
 \& use} 

\end{frame}
}


\subsection{Installing}

\begin{frame}
\frametitle{Installing Python}

\begin{description}

\item<1->[Hard way:] download source and compile:\\
\url{https://www.python.org/downloads/}

\item<2->[Normal way:] use installer or package:

\begin{itemize}
\item Windows: \url{https://www.python.org/downloads/windows/}
\item Mac OS: \url{https://www.python.org/downloads/mac-osx/}
\item Linux: package manager:\\
python3.x and python3.x-dev packages
\end{itemize}

\item<3->[Easy way:] Python distributions such as:\\
\href{https://www.anaconda.com/distribution/}{Anaconda} \comment{free}\\
\href{https://assets.enthought.com/downloads/}{Enthought Canopy} \comment{free and commercial}\\
\href{http://python-xy.github.io/}{Python(x,y)} \comment{free, Windows only}\\
+ others

\end{description}

\end{frame}

%--------------------------------------------------------------------------------------------------------------
\begin{frame}[fragile]
\frametitle{Installing modules~\faWrench}

Example: SciPy (\url{https://www.scipy.org/install.html}):\\
mathematics, science, and engineering

\begin{description}
\item<2->[Easy way:] Windows, Linux, Mac:\\
use Scientific Python distribution\comment{(ex: Anaconda)}
\item<3->[Intermediate:] Linux, Mac: install package\\
\begin{itemize}
\item Linux: package manager
\item Mac: \href{https://www.macports.org/}{MacPorts}, \href{http://brew.sh/}{Homebrew}
\end{itemize}
\item<4->[Harder:] build from source
\begin{lstlisting}[language=bash]
python setup.py install
\end{lstlisting}
\item<5->[Better:] pip (\url{https://pypi.python.org/pypi/pip})
\item<6->[\faWarning~Avoid:] mixing installation methods

\end{description}

\end{frame}


%--------------------------------------------------------------------------------------------------------------
\begin{frame}[fragile]
\frametitle{Using \texttt{pip} to manage modules}

\texttt{pip} = recommended tool for installing Python packages

\begin{description}
\item<2->[Installation:]
\begin{itemize}
\item Included in recent Python version
\item Otherwise: download and run \pyfile{get-pip.py}\\
\url{https://pip.pypa.io/en/stable/installing/\#install-pip}
\begin{lstlisting}[language=bash]
python get-pip.py
\end{lstlisting}
\end{itemize}

\item<3->[Usage:]
\begin{itemize}
\item Install latest version + dependencies:
\begin{lstlisting}[language=bash]
pip install PackageName
\end{lstlisting}

\item Specify exact version:
\begin{lstlisting}[language=bash]
pip install PackageName==x.y.z
\end{lstlisting}
\item Specify minimum version:
\begin{lstlisting}[language=bash]
pip install 'PackageName>=x.y.z'
\end{lstlisting}
\end{itemize}

\end{description}

\end{frame}

%--------------------------------------------------------------------------------------------------------------
\begin{frame}[t, fragile]
\frametitle{More about \texttt{pip}}


\begin{itemize}
\item<1-> Uninstall packages:                   
\begin{lstlisting}[language=bash]
pip uninstall PackageName
\end{lstlisting}
\item<2->List installed packages:
\begin{lstlisting}[language=bash]
pip list
\end{lstlisting}

\begin{onlyenv}<2>
Output:
\tiny
\begin{lstlisting}[language=bash]
Package          Version    
---------------- -----------
asn1crypto       0.24.0     
Cartopy          0.17.0     
certifi          2019.6.16  
cffi             1.12.3    
...
\end{lstlisting}
\end{onlyenv}


\item<3-> List installed packages in requirements format:
\begin{lstlisting}[language=bash]
pip freeze
\end{lstlisting}

\begin{onlyenv}<3>
Output:
\tiny
\begin{lstlisting}[language=bash]
asn1crypto==0.24.0
Cartopy==0.17.0
...
tornado==6.0.3
urllib3==1.25.3
\end{lstlisting}
\end{onlyenv}
\end{itemize}
\end{frame}


%--------------------------------------------------------------------------------------------------------------
\begin{frame}[t, fragile]
\frametitle{More about \texttt{pip}}

\begin{itemize}
\item<1-> Show information about installed packages:
\begin{lstlisting}[language=bash]
pip show PackageName
\end{lstlisting}

\begin{onlyenv}<1>
\tiny
\begin{lstlisting}[language=bash]
ctroupin@GHER-ULg-Laptop ~ $ pip show numpy
Name: numpy
Version: 1.17.1
Summary: NumPy is the fundamental package for array computing with Python.
Home-page: https://www.numpy.org
Author: Travis E. Oliphant et al.
Author-email: None
License: BSD
Location: /home/ctroupin/miniconda3/lib/python3.7/site-packages
Requires: 
Required-by: scipy, pykdtree, matplotlib, Cartopy
\end{lstlisting}
\end{onlyenv}

\end{itemize}

\end{frame}

%--------------------------------------------------------------------------------------------------------------

\begin{frame}[c, fragile]
\frametitle{Exercise 1: changecase.py}

\exercise

Write a program that takes 2 arguments: the name and the surname, both written with a mix of upper and lowercase, and return the name with the first letter in uppercase and the surname with all the letters in uppercase.

\textbf{Examples:}

\textit{changecase allan rickman} \hspace{1cm} returns \hspace{1cm}  \textit{Allan RICKMAN} \\
\textit{changecase aLlAn ricKmaN} \hspace{1cm} returns \hspace{1cm}  \textit{Allan RICKMAN} 

\vspace{1cm}

\textbf{Tips:} use the function \href{https://docs.python.org/2/library/sys.html#sys.argv}{sys.argv} to be able to run the code as 
\begin{lstlisting}[language=bash]
changecase name surname
\end{lstlisting}


\end{frame}

